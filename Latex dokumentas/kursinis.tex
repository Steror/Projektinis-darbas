\documentclass{VUMIFPSkursinis}
\usepackage{algorithmicx}
\usepackage{algorithm}
\usepackage{algpseudocode}
\usepackage{amsfonts}
\usepackage{amsmath}
\usepackage{bm}
\usepackage{caption}
\usepackage{color}
\usepackage{float}
\usepackage{graphicx}
\usepackage{listings}
\usepackage{subfig}
\usepackage{wrapfig}
\usepackage{longtable}

\usepackage{enumitem}
%PAKEISTA, tarpai tarp sąrašo elementų
\setitemize{noitemsep,topsep=0pt,parsep=0pt,partopsep=0pt}
\setenumerate{noitemsep,topsep=0pt,parsep=0pt,partopsep=0pt}

% Titulinio aprašas
\university{Vilniaus universitetas}
\faculty{Matematikos ir informatikos fakultetas}
\department{Programų sistemų katedra}
\papertype{Kursinis darbas}
\title{Paieškos proceso ir jos rezultatų pateikimo vartotojams panaudojamumas VUL Santaros klinikų puslapyje}
\titleineng{The Usability of the Search Process and Presenting its Results to the User for VUH Santaros klinikos website}
\status{4 kurso 3 grupės studentas}
\author{Tomas Kiziela}
% \secondauthor{Vardonis Pavardonis}   % Pridėti antrą autorių
\supervisor{doc. Kristina Lapin}
\date{Vilnius – \the\year}

% Nustatymai
% \setmainfont{Palemonas}   % Pakeisti teksto šriftą į Palemonas (turi būti įdiegtas sistemoje)
\bibliography{bibliografija}

\begin{document}
	
% PAKEISTA	
\maketitle
\cleardoublepage\pagenumbering{arabic}
\setcounter{page}{2}

%TURINYS
\tableofcontents



\sectionnonum{Įvadas}
%Įvade apibūdinamas darbo tikslas, temos aktualumas ir siekiami rezultatai.
%Darbo įvadas neturi būti dėstymo santrauka. Įvado apimtis 1–2 puslapiai.


Uždaviniai:
\begin{enumerate}
	\item Identifikuoti technologijas reikalingas puslapio kūrimui
	\item Išskirti lyginimo kriterijus remiantis literatūros šaltiniais
	\item Sukurti galutinio sprendimo prototipą
	%\item Paruošti sprendimo variantus
	%\renewcommand*{\theenumii}{\theenumi.\arabic{enumii}}
	%\renewcommand{\labelenumii}{\theenumii}
	%\begin{enumerate}
	%	\item Remiantis lyginimo kriterijais ir literatūros šaltiniais išskirti alternatyvius sprendimus
	%	\item Sukurti sprendimo variantų maketus
	%	\item Palyginti maketus
	%	\item Sukurti galutinio sprendimo prototipą
	%\end{enumerate}
	%\item Išskirti detalius reikalavimus galutinio sprendimo įgyvendinimui
	%\item Atlikus literatūros analizę išskirti technologijas padedančias įgyvendinti galutinį sprendimą
\end{enumerate}

\section{Sistemos architektūros modelis}
Prieš kuriant internetinį tinklapį reikia apgalvoti, kokia bus sistemos archtektūra. Tai nulemia naudotojų poreikiai, įgyvendinimo sudėtingumas, populiarūs sprendimai.

Vienas iš populiariausių ir paprasčiausių architektūros modelių yra Modelis-Vaizdas-Valdiklis (Model-View-Controller, toliau MVC)\cite{MVCDefinition}. Šis modelis sudarytas iš trijų sluoksnių: duomenų sluoksnio, vaizdo sluoksnio ir valdiklių sluoksnio. Duomenų sluoksnis atsakingas už duomenis reikalingus programos veikimui, pavyzdžiui duomenų bazę. Vaizdo sluoksnis pateikia vartotojui vaizdą, pavyzdžiui puslapį ir mygtukus. Valdiklių sluoksnis skirtas komunikacijai tarp vaizdo ir modelio sluoksnių, jis priima vartotojo įvestį ir pateikia rezultatus iš duomenų bazės. Visa tai leidžia atlikti tinklapiui reikalingas funkcijas kaip duomenų, saugojimas, puslapių rodymas, paieškų atlikimas, filtravimas ir žinučių ar komentarų siuntimas. Naudojant MVC modelį kodas atskiriamas pagal sluoksnius, tai leidžia izoliuoti komponentus ir dėl to kodą lengviau plėsti, kyla mažiau klaidų bei jas lengviau pataisyti.

Kadangi tinklapiui nėra griežtų ir specifinių reikalavimų, MVC modelis tinka sistemos kūrimui\cite{MVCSO1, MVCSO2}. Vienas svarbus punktas yra, kad sistema būtų responsyvi ir galėtų aptarnauti didelį kiekį vartotojų, tačiau MVC modelis nesukelia tam problemų.

\section{Sistemą realizuojančios technologijos}

%Medžiagos darbo tema dėstymo skyriuose pateikiamos nagrinėjamos temos detalės:
%pradinė medžiaga, jos analizės ir apdorojimo metodai, sprendimų įgyvendinimas,
%gautų rezultatų apibendrinimas. Šios dalies turinys labai priklauso nuo darbo
%temos. Skyriai gali turėti poskyrius ir smulkesnes sudėtines dalis, kaip
%punktus ir papunkčius.

%Medžiaga turi būti dėstoma aiškiai, pateikiant argumentus. Tekstas dėstomas
%trečiuoju asmeniu, t.y. rašoma ne „aš manau“, bet „autorius mano“, „autoriaus
%nuomone“. Reikėtų vengti informacijos nesuteikiančių frazių, pvz., „...kaip jau
%buvo minėta...“, „...kaip visiems žinoma...“ ir pan., vengti grožinės literatūros
%ar publicistinio stiliaus, gausių metaforų ar panašių meninės išraiškos
%priemonių.



%\section{Reikalavimai galutinio sprendimo įgyvendinimui}
%
%Norint sukurti galutinį sprendimą liko išspręsti dar keletą problemų, šiame skyriuje nagrinėjamos šios problemos. 

%\section{Technologijos galutinio sprendimo įgyvendinimui}

\sectionnonum{Rezultatai ir išvados}


%Rezultatų ir išvadų dalyje turi būti aiškiai išdėstomi pagrindiniai darbo
%rezultatai (kažkas išanalizuota, kažkas sukurta, kažkas įdiegta) ir pateikiamos
%išvados (daromi nagrinėtų problemų sprendimo metodų palyginimai, teikiamos
%rekomendacijos, akcentuojamos naujovės).


%% PAKEISTAS PAVADINIMAS Į 'Šaltiniai'
\printbibliography[heading=bibintoc, title=Šaltiniai]  % Šaltinių sąraše nurodoma panaudota
% literatūra, kitokie šaltiniai. Abėcėlės tvarka išdėstomi darbe panaudotų
% (cituotų, perfrazuotų ar bent paminėtų) mokslo leidinių, kitokių publikacijų
% bibliografiniai aprašai.  Šaltinių sąrašas spausdinamas iš naujo puslapio.
% Aprašai pateikiami netransliteruoti. Šaltinių sąraše negali būti tokių
% šaltinių, kurie nebuvo paminėti tekste.

% \sectionnonum{Sąvokų apibrėžimai}
%\sectionnonum{Santrumpos}
%Sąvokų apibrėžimai ir santrumpų sąrašas sudaromas tada, kai darbo tekste
%vartojami specialūs paaiškinimo reikalaujantys terminai ir rečiau sutinkamos
%santrumpos.

%\appendix  % Priedai
% Prieduose gali būti pateikiama pagalbinė, ypač darbo autoriaus savarankiškai
% parengta, medžiaga. Savarankiški priedai gali būti pateikiami ir
% kompaktiniame diske. Priedai taip pat numeruojami ir vadinami. Darbo tekstas
% su priedais susiejamas nuorodomis.






%\section{Eksperimentinio palyginimo rezultatai}
% tablesgenerator.com - converts calculators (e.g. excel) tables to LaTeX
%\begin{table}[H]\footnotesize
%  \centering
%  \caption{Lentelės pavyzdys}
%  {\begin{tabular}{|l|c|c|} \hline
%    Algoritmas & $\bar{x}$ & $\sigma^{2}$ \\
%    \hline
%    Algoritmas A  & 1.6335    & 0.5584       \\
%    Algoritmas B  & 1.7395    & 0.5647       \\
%    \hline
%  \end{tabular}}
%  \label{tab:table example}
%\end{table}

\end{document}
